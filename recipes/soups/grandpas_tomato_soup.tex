\documentclass{article}
\usepackage{cuisine}
\begin{document}
\begin{recipe}{Grandpa's tomato soup}{8 portions}{3 hours}
\ingredient[1]{}{pork bone}
\ingredient[3]{}{porkchops}
\ingredient[3]{l}{water}
Wash the bone and the meat. Put them into a tall pot and cover with water.
Begin boiling the stock.

\ingredient[3]{}{carrots}
\ingredient[1]{}{parsley}
\ingredient[1]{}{celery}
\ingredient[1]{}{onion}
Wash and peel the vegetables. Cut the celery in half, and the carrots and parsley into four pieces.
Cut the onion in half and fry without oil until golden. Add the ingredients to the soup. 

\ingredient[1]{tbsp}{salt}
\ingredient[7]{}{peppercorns}
\ingredient[5]{}{allspice}
\ingredient[5]{}{bay leaves}
Add the spices to the broth. Cook for at least two hours on very low heat.

\ingredient[90]{g}{tomato concentrate}
\ingredient[]{}{sour cream}
Strain the soup, dispose of everything aside from the carrots and the porkchops. Add the concentrate and
carrots to the broth. Add water to obtain three liters if necessary. Cook for ten more minutes. 
Serve with chopped meat and sour cream.

\end{recipe}
\end{document}
